\documentclass[a4paper,11pt]{article}
\usepackage[utf8]{inputenc}
\usepackage[spanish]{babel}
\usepackage[hmargin=3cm, vmargin=3cm]{geometry}
\usepackage{graphicx}
\usepackage{amssymb}
\usepackage{amsmath}
\usepackage{enumitem}
\usepackage{subcaption}
\usepackage{fancyhdr}
\usepackage{titlesec}


\titleformat{\section}
  {\bf}{Problema \thesection.}{0.5em}{}


%%%%%%%%%%%%%%%%%%%%%%%%%%%%%%%%%%%%%%%%%%%%%%%%%%%%%%%%%%%%%%%%%%%%%%%%%%%%%%

\begin{document}


% Fancy Header
% ------------
\pagestyle{fancy}
%~ \renewcommand{\headrulewidth}{0pt}
\lhead{\small Veronica Gargiulo}
\chead{\small \the\year}
\rhead{\small Santiago Soler}



% Title
% -----
\thispagestyle{plain}
\begin{center}
    \textbf{\large
        Mecánica Estadística \\
        Práctica 2 - Colectivo Canónico
    }
\end{center}
\vspace{-1.5em}



% Excersises
% ----------

\section{Gas Ideal Clásico}
\label{sec:gas-ideal}

La forma más sencilla de modelizar un gas real es el \emph{gas ideal 
clásico}, el cual consiste en considerar un conjunto de $N$ partículas 
puntuales cuyas únicas interacciones son colisiones 
perfectamente elásticas.
El caracter clásico del modelo hace referencia a que la dinámica de 
las $N$ partículas puede ser abordada desde la mecánica clásica.

En este problema vamos a considerar un gas ideal dentro de un 
recipiente de volumen $V$ y en contacto con un baño térmico de 
temperatura $T$, de tal manera que el recipiente permite intercambiar 
energía con el baño, y el sistema ``baño + recipiente'' se encuentra 
aislado.

\begin{enumerate}[label=(\alph*),
                  leftmargin=2\parindent,
                  rightmargin=2\parindent]

    \item{\label{item:gas-ideal-particion}
          Si todas las partículas del gas poseen la misma masa $m$, 
          mostrar que la función de partición del gas ideal puede ser:
          $$ Z(N, T, V) = V^N \left[ 2\pi m k_B T \right]^{3N/2} $$
          }

    \item{Calcular la energía libre de Helmholtz y deducir la ecuación de
          estado del gas ideal:
          $$ pV = N k_B T $$
          }
    
    \item{Calcular el valor medio de la energía del sistema y 
          verificar si se cumple el teorema de equipartición.
          }

    \item{Calcular la entropía y los calores específicos $C_V$ y $C_p$.
          }
    
    \item{¿Es la entropía obtenida en el punto anterior una propiedad 
          extensiva?
          }
    
    {\small
    \textbf{Ayuda:} Verificar que la suma de las entropías de dos 
    gases $A$ y $B$ es igual a la entropía del sistema $A + B$.
    }
    
    \item{Agregar la corrección $1/N!$ a la función de partición y 
          verificar si la entropía obtenida a partir de ella es extensiva.
          $$ Z(N, T, V) = \frac{V^N}{N!} \left[ 2\pi m k_B T \right]^{3N/2} $$
          ¿Cómo podemos interpretar la corrección $1/N!$ en función 
          de la distinguibilidad de las partículas clásicas?
          }

\end{enumerate}


\section{Longitud de onda de de Broglie}

Consideremos el gas ideal clásico del Problema \ref{sec:gas-ideal} y 
repensemos el modelo de partículas clásicas no interactuantes. Para que esta 
simplificación sea válida es necesario basarnos en la hipótesis de que las 
partículas del gas se encuentran a una distancia suficientemente grande en la 
que las funciones de onda de cada una de ellas no se superpone con las del 
resto, es decir, a una distancia en la cual podemos obviar la coherencia 
cuántica.

Al calcular la función de partición hemos integrado el factor 
$e^{-\beta E}$ en el espacio de las fases. Sin embargo no hemos tenido 
en cuenta un concepto de la mecánica cuántica: el principio de 
incertidumbre de Heisenberg. El mismo dicta que nos es imposible medir 
con precisión ciertos pares de magnitudes físicas de un sistema 
cuántico, por ejemplo, la posición y el momento. Este principio se 
puede resumir en:
$$ \Delta x \Delta p \gtrsim  h$$
donde $\Delta x$ y $\Delta p$ son las incertidumbres en la medición de 
la posición y el momento de una particula, respectivamente; y $h$ es 
la constante de Planck.
De esta manera, podríamos pensar que a la hora de integrar en las 
coordenadas del espacio de las fases es necesario dividir por el 
``mínimo volumen'' que puede ocupar cada microestado 
$\{ q_1, \dots q_N, p_1, \dots p_N \}$, es decir, $h^{3N}$.

\begin{enumerate}[label=(\alph*),
                  leftmargin=2\parindent,
                  rightmargin=2\parindent]
      
    \item{Obtener la función de partición a partir de la siguiente 
          integral:
          $$
          Z(N, T, V) =
            \frac{1}{h^{3N} N!}
            \int_D e^{-\beta E(\{q, p\})} d^{3N}q \,\, d^{3N}p
          $$
          }

    \item{Definimos a la longitud de onda térmica de de Broglie como:
          $$ \lambda = \left( \frac{h^2}{2\pi m k_B T} \right)^{1/2} $$
          Expresar la función de partición obtenida en el punto 
          anterior en función de $\lambda$.
          }

    \item{La longitud de onda de de Broglie se puede considerar como la 
          longitud característica a partir de la cual la coherencia 
          cuántica comienza a tener relevancia.\\
          Obtener una expresión para la distancia media de las 
          partículas ($l$). ¿Cómo deberíamos considerar a las 
          partículas si $\lambda/l \ll 1$? ¿y en el caso contrario?
          }
          
\end{enumerate}


\section{Paradoja de Gibbs}

En el Problema \ref{sec:gas-ideal} hemos analizado cómo la corrección $1/N!$ 
permite que los potenciales termodinámicos sean extensivos. En este problema 
veremos que ésta además soluciona otro problema: la paradoja de Gibbs.

Consideremos un sistema aislado compuesto por dos gases $A$ y $B$ en 
equilibrio térmico que se encuentran separados por un tabique que 
permite intercambiar energía entre ellos.
Los subsistemas $A$ y $B$ constan de $N_A$ y $N_B$ partículas del mismo 
tipo y volúmenes $V_A$ y $V_B$, respectivamente.
Luego, retiramos el tabique permitiendo que ambos gases se mezclen.

\begin{figure}[h!]
    \centering
    \begin{subfigure}[b]{0.3\textwidth}
        \includegraphics[width=\textwidth]{figs/paradoja-gibbs-1.png}
        \caption{Estado inicial del sistema}
        \label{fig:gibbs-tabique}
    \end{subfigure}
    \hspace{0.1\textwidth}
    \begin{subfigure}[b]{0.3\textwidth}
        \includegraphics[width=\textwidth]{figs/paradoja-gibbs-2.png}
        \caption{Estado final del sistema}
        \label{fig:gibbs-mezcla}
    \end{subfigure}
    \caption{Paradoja de Gibbs en mezcla de gases}
\end{figure}

A partir de un análisis puramente termodinámico podemos afirmar que:

\begin{itemize}

\item{Si ambos gases poseen densidades diferentes
      $\left( \frac{N_A}{V_A} \neq \frac{N_B}{V_B} \right)$, entonces la 
      mezcla de los gases es un proceso irreversible y por ende
      $ \Delta S > 0 $.
      }

\item{Si por el contrario, ambos gases poseen la misma densidad
      $\left( \frac{N_A}{V_A} = \frac{N_B}{V_B} \right)$, entonces 
      el proceso es reversible y $ \Delta S = 0 $.
      }

\end{itemize}

\begin{enumerate}[label=(\alph*),
                  leftmargin=2\parindent,
                  rightmargin=2\parindent]

    \item{Calcular la diferencia entre la entropía final e inicial del 
          sistema sin aplicar la corrección $1/N!$.
          Analizar si el signo de la variación de la entropía 
          concuerda con el análisis termodinámico para gases $A$ y $B$ de 
          igual y de distinta densidad.}
    
    \item{Calcular la misma variación de entropía ahora con la 
          corrección $1/N!$. ¿Dicha corrección resuelve la paradoja 
          de Gibbs?}

\end{enumerate}


\section{Gas ideal clásico: niveles de energía discretos}

En los problemas anteriores hemos observado que un desarrollo mecánico 
estadístico de gases ideales puramente clásicos posee problemas 
conceptuales que deben resolverse a través de agregados provenientes 
de razonamientos cuánticos.
Si bien una completa descripción cuántica del mismo involucra 
simetrizaciones de la función de onda (que veremos más adelante), en 
este problema intentaremos incorporar algunos aspectos mecánico 
cuánticos.

Una forma de aproximar un gas ideal es considerarlo como un conjunto de 
$N$ partículas cuánticas no interactuantes, cuya distancia con el resto 
es lo suficientemente grande como para poder obviar la coherencia 
cuántica. Dado que las partículas se encuentran encerradas en un 
recipiente de volumen $V$, podemos considerar a cada partícula inmersa 
en un pozo de potencial infinito de ancho $V^{1/3}$ en cada dirección 
($x, y, z$).

Los niveles energía de una partícula en un pozo de potencial infinito 
unidimensional vienen dados por:
$$ E_n = \frac{\hbar^2 \pi^2}{2m V^{2/3}} n^2, \quad n=1, 2, 3, \dots $$

\noindent Mientras que la misma partícula en un pozo de potencial infinito 
tridimensional posee el siguiente espectro de energía:
$$ E_{n_x, n_y, n_z} = \frac{\hbar^2 \pi^2}{2m V^{2/3}} (n_x^2 + n_y^2 + n_z^2),
\quad n_x, n_y, n_z = 1, 2, 3, \dots $$


Además de suponer que las partículas no interactúan entre sí, 
asumiremos que son indistinguibles, lo que nos permite expresar la 
función de partición del sistema como:
$$ Z(N,T,V) = \frac{1}{N!} [Z_1(T, V)]^N $$
donde $Z_1$ es la función de partición de una sola partícula.


\begin{enumerate}[label=(\alph*),
                  leftmargin=2\parindent,
                  rightmargin=2\parindent]

    \item{Calcular la función de partición del sistema.}
    
    {\small
    \textbf{Ayuda:} Puede ser necesario aproximar la sumatoria 
    discreta de los estados en una integral del tipo:
    $$ Z_1(T, V) =
    \int_0^\infty g(\epsilon) e^{-\epsilon/k_B T} d\epsilon $$
    donde $g(e)$ es la densidad de estados y puede calcularse como:
    $$ g(\epsilon) = \frac{d N(\epsilon)}{d\epsilon} $$
    donde $N(\epsilon)$ es la cantidad de estados con energía menor o 
    igual a $\epsilon$.
    }

\end{enumerate}


\end{document}

