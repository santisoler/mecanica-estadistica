\documentclass[a4paper,11pt]{article}
\usepackage[utf8]{inputenc}
\usepackage[spanish]{babel}
\usepackage[hmargin=3cm, vmargin=3cm]{geometry}
\usepackage{graphicx}
\usepackage{amssymb}
\usepackage{amsmath}
\usepackage{enumitem}
\usepackage{subcaption}
\usepackage{fancyhdr}
\usepackage{titlesec}


\titleformat{\section}
  {\bf}{Problema \thesection.}{0.5em}{}


%%%%%%%%%%%%%%%%%%%%%%%%%%%%%%%%%%%%%%%%%%%%%%%%%%%%%%%%%%%%%%%%%%%%%%%%%%%%%%

\begin{document}


% Fancy Header
% ------------
\pagestyle{fancy}
%~ \renewcommand{\headrulewidth}{0pt}
\lhead{\small Verónica Gargiulo}
\chead{\small \the\year}
\rhead{\small Santiago Soler}



% Title
% -----
\thispagestyle{plain}
\begin{center}
    \textbf{\large
        Mecánica Estadística \\
        Práctica 1 - Repaso de Termodinámica y Análisis Combinatorio
    }
\end{center}
\vspace{-1.5em}



% Excersises
% ----------

\section{}

Entre los años 1995 y 2016 la Argentina adoptó un sistema de patentamiento de
automóviles que contaba con la combinación de 3 letras y 3 números (ej.~ABC~123).
A partir de 2016 el sistema se modificó agregando una letra adicional (ej.~AB~123~CD).

\begin{enumerate}[label=(\alph*),
                  leftmargin=2\parindent,
                  rightmargin=2\parindent]

    \item{
        Calcule la cantidad total de autos que podrían registrase bajo el viejo sistema
        si a cada uno de ellos le corresponde una única patente.
    }

    \item{
        Calcule la cantidad total de autos que se pueden registrar bajo el nuevo
        sistema.
    }

    \item{
        ¿Cuántos autos tendrán patentes (del nuevo sistema) con solo vocales en el
        primer par de letras, solo números pares (incluido el cero) y solo consonantes
        en las últimas dos letras (ej.~AE~248~FG)?
    }

\end{enumerate}


\section{}

\begin{enumerate}[label=(\alph*),
                  leftmargin=2\parindent,
                  rightmargin=2\parindent]

    \item{
        ¿De cuántas formas es posible ordenar los símbolos
        \emph{a, b, c, d, e, f, g, h, i}?
    }

    \item{
        Si reemplazamos los últimos cuatro símbolos por \emph{e},
        ¿De cuántas formas es posible ordenar el conjunto
        \emph{a, b, c, d, e, e, e, e, e} (tenga en cuenta que cada \emph{e} es
        indistinguible de las demás)?
    }

    \item{
        ¿De cuántas formas es posible ordenar el conjunto del item anterior de tal
        manera que ninguna \emph{e} quede junto a otra?
    }

\end{enumerate}


\section{}

¿De cuántas formas se puede formar un equipo de baloncesto de cinco personas con 12
posibles jugadores? ¿Cuántas opciones incluyen al jugador más alto y al más bajo
simultáneamente?


\section{}

Un comité de 12 personas será elegido entre 10 hombres y 10 mujeres. ¿De cuántas formas
se puede hacer la selección si:
(a) no hay restricciones?
(b) debe haber seis hombres y seis mujeres?


\section{}

¿De cuántas formas es posible distribuir 10 monedas idénticas entre cinco niños si:
(a) no hay restricciones?
(b) cada niño recibe al menos una moneda?
(c) el niño mayor recibe al menos dos monedas?

\section{}

Una tienda de helados tiene disponibles 13 sabores de helado. ¿De cuántas formas se
puede comprar una docena de conos de helado si:
(a) no queremos el mismo sabor más de una vez?
(b) un mismo sabor puede pedirse cuantas veces como se desee?


\section{}

¿De cuántas formas se pueden distribuir ocho bolas blancas idénticas en cuatro
recipientes distintos de modo que:
(a) ningún recipiente quede vacío?
(b) el cuarto recipiente contenga un número impar de bolas?


\section{}

Consideremos $n$ moles de un gas ideal monoatómico que se encuentra en
un estado inicial de presión $P_A$ y volumen $V_A$. Supongamos que se
incrementa la temperatura a volumen constante hasta duplicar la
presión. Luego el gas se expande isotérmicamente hasta que la presión
desciende a su valor original y posteriormente se comprime a presión
constante hasta que el volumen recupera su valor inicial.

\begin{enumerate}[label=(\alph*),
                  leftmargin=2\parindent,
                  rightmargin=2\parindent]

    \item{Representar estos procesos en el plano $P$-$V$ y en el plano
          $P$-$T$.}

    \item{Calcular el trabajo realizado por el sistema, el calor
          entregado al mismo y su variación de la energía interna en
          cada proceso.}

    \item{Calcular la variación energía interna del ciclo completo e
          interpretar el resultado.}

    \item{Calcular el trabajo realizado por el sistema y el calor
          entregado al mismo a lo largo del ciclo completo. ¿El sistema
          realiza trabajo o se realiza trabajo sobre el mismo? ¿El
          sistema recibe calor o lo entrega al entorno?}

\end{enumerate}


\section{}

Consideremos dos bloques idénticos $A$ y $B$ que se encuentran
inicialmente a temperaturas $T_A$ y $T_B$, respectivamente, tal que
$T_A > T_B$. Dichos bloques se ponen en contacto dentro de un
recipiente con paredes adiabáticas hasta que ambos alcanzan la misma
temperatura $T_f$.

\begin{enumerate}[label=(\alph*),
                  leftmargin=2\parindent,
                  rightmargin=2\parindent]

    \item{Analizando cualitativamente el proceso que sufren ambos
          bloques, ¿será reversible?}

    \item{Si el calor específico de cada bloque es $C_V = 3 N k_B$,
          donde $N = 10^{23}$ y $k_B$ es la constante de Boltzmann,
          calcular la variación de entropía de cada bloque a lo largo
          del proceso y la del sistema $A \cup B$.
          Teniendo en cuenta este último resultado, ¿el proceso es
          reversible?}

\end{enumerate}


\section{}

Consideremos un diamante monocristalino de 10g que ha sido utilizado
como elemento de corte, proceso que ha elevado su temperatura a
300$^\circ$C.
Con el objeto de enfriarlo, se lo sumerge en un baño de agua a
10$^\circ$C hasta que alcanza el equilibrio termodinámico con el mismo.
El diamante es un material cuyo calor específico a temperatura
ambiente depende de $T$ a través de la siguiente expresión:

$$ C_V(T) = \frac{12\pi^4}{5} N k_B \left( \frac{T}{\Theta_D} \right)^3, $$

\noindent donde $N$ es la cantidad de átomos de carbono que componen
al diamante, $k_B$ es la constante de Boltzmann y $\Theta_D$ es la
temperatura de Debye, que en el caso del diamante asume un valor de
$\Theta_D = 1860$K.

\begin{enumerate}[label=(\alph*),
                  leftmargin=2\parindent,
                  rightmargin=2\parindent]

    \item{\label{item:calor-diamante}
          Calcular el calor que transmite el diamante hacia el baño
          térmico durante el enfriado.}

    \item{Determinar la variación de entropía del diamante.}

    \item{Utilizando el resultado del item~\ref{item:calor-diamante},
          estimar la variación de entropía del baño y luego el del sistema
          ``baño + diamante''. ¿El enfriamiento del diamante es
          reversible?
          }

    {\small
    \textbf{Ayuda:}
    La masa atómica del C es de 12.01 g/mol.
    Podemos considerar que el baño de agua es tan grande que no sufre
    variaciones apreciables de temperatura durante el enfriado.
    }

\end{enumerate}

\end{document}

