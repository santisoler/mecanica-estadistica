\documentclass[a4paper,11pt]{article}
\usepackage[utf8]{inputenc}
\usepackage[spanish]{babel}
\usepackage[hmargin=3cm, vmargin=3cm]{geometry}
\usepackage{graphicx}
\usepackage{amssymb}
\usepackage{amsmath}
\usepackage{braket}
\usepackage{enumitem}
\usepackage{subcaption}
\usepackage{fancyhdr}
\usepackage{titlesec}


\titleformat{\section}
  {\bf}{Problema \thesection.}{0.5em}{}


%%%%%%%%%%%%%%%%%%%%%%%%%%%%%%%%%%%%%%%%%%%%%%%%%%%%%%%%%%%%%%%%%%%%%%%%%%%%%%

\begin{document}


% Fancy Header
% ------------
\pagestyle{fancy}
%~ \renewcommand{\headrulewidth}{0pt}
\lhead{\small Graciela Gómez}
\chead{\small \the\year}
\rhead{\small Santiago Soler}



% Title
% -----
\thispagestyle{plain}
\begin{center}
    \textbf{\large
        Mecánica Estadística \\
        Práctica 3 - Colectividad Canónica
    }
\end{center}
\vspace{-1.5em}



% Excersises
% ----------

\section{}

Considere una partícula que puede acceder a dos niveles de energía ($\epsilon_0 = 0$eV
y $\epsilon_1 = 0.01$eV) situada en un baño térmico a una temperatura $T$.

\begin{enumerate}[label=(\alph*),
                  leftmargin=2\parindent,
                  rightmargin=2\parindent]

    \item{Calcule la función de partición de dicha partícula.}

    \item{Determine las probabilidades de hallar a la partícula en el nivel $\epsilon_0$
          y $\epsilon_1$ en caso de que $T =$ 1K, 100K, 300K y 1000K.
          Para obtener mayor información, grafique dichas probabilidades en función de
          $T$.
          Interprete estos resultados microscópicamente.}

    \item{Calcule la función de partición y las probabilidades del item anterior en caso
          de que el nivel fundamental posea degeneración 2.
          ¿Qué diferencias halla con el caso anterior? ¿Cómo se pueden comprender?}

\end{enumerate}

\section{Factorización de la Función de Partición}

\begin{enumerate}[label=(\alph*),
                  leftmargin=2\parindent,
                  rightmargin=2\parindent]

    \item{Dado un sistema compuesto por dos subsistemas no interactuantes entre sí,
          demuestre que la función de partición del gran sistema se puede expresar
          como el producto de las funciones de particiones de cada subsistema:
          $$ Z(N, V, T) = Z_A(N_A, V_A, T) Z_B(N_B, V_B, T) $$
          }

    \item{Dado un sistema compuesto por $N$ partículas no interactuantes que poseen los
          mismos niveles energéticos monoparticulares, demostrar que su función de
          partición puede ser factorizada entre la función de partición de cada
          partícula:
          $$ Z(N, V, T) = [Z_1(V, T)]^N $$
          }

\end{enumerate}


\section{Gas ideal clásico}
\label{sec:gas-ideal}

La forma más sencilla de modelizar un gas real es el \emph{gas ideal
clásico}, el cual consiste en considerar un conjunto de $N$ partículas
puntuales cuyas únicas interacciones son colisiones
perfectamente elásticas.
El caracter clásico del modelo hace referencia a que la dinámica de
las $N$ partículas puede ser abordada desde la mecánica clásica.

En este problema vamos a considerar un gas ideal dentro de un
recipiente de volumen $V$ y en contacto con un baño térmico de
temperatura $T$, de tal manera que el recipiente permite intercambiar
energía con el baño, y el sistema ``baño + recipiente'' se encuentra
aislado.

\begin{enumerate}[label=(\alph*),
                  leftmargin=2\parindent,
                  rightmargin=2\parindent]

    \item{\label{item:gas-ideal-particion}
          Si todas las partículas del gas poseen la misma masa $m$,
          mostrar que la función de partición del gas ideal puede ser:
          $$ Z(N, T, V) = V^N \left[ 2\pi m k_B T \right]^{3N/2} $$
          }

    \item{Calcular la energía libre de Helmholtz y deducir la ecuación de
          estado del gas ideal:
          $$ pV = N k_B T $$
          }

    \item{Calcular el valor medio de la energía del sistema y
          verificar si se cumple el teorema de equipartición.
          }

    \item{Calcular la entropía y los calores específicos $C_V$ y $C_p$.
          }

    \item{¿Es la entropía obtenida en el punto anterior una propiedad
          extensiva?
          }

    {\small
    \textbf{Ayuda:} Verificar que la suma de las entropías de dos
    gases $A$ y $B$ es igual a la entropía del sistema $A + B$.
    }

\end{enumerate}


\section{Longitud de onda de de Broglie}
\label{sec:de-broglie}

Consideremos el gas ideal clásico del Problema \ref{sec:gas-ideal} y
repensemos el modelo de partículas clásicas no interactuantes. Para que esta
simplificación sea válida es necesario basarnos en la hipótesis de que las
partículas del gas se encuentran a una distancia suficientemente grande en la
que las funciones de onda de cada una de ellas no se superpone con las del
resto, es decir, a una distancia en la cual podemos obviar la coherencia
cuántica.

Al calcular la función de partición hemos integrado el factor
$e^{-\beta E}$ en el espacio de las fases. Sin embargo no hemos tenido
en cuenta un concepto de la mecánica cuántica: el principio de
incertidumbre de Heisenberg. El mismo dicta que nos es imposible medir
con precisión ciertos pares de magnitudes físicas de un sistema
cuántico, por ejemplo, la posición y el momento. Este principio se
puede resumir en:
$$ \Delta x \Delta p \gtrsim  h$$
donde $\Delta x$ y $\Delta p$ son las incertidumbres en la medición de
la posición y el momento de una particula, respectivamente; y $h$ es
la constante de Planck.
De esta manera, cada microestado
$\{ q_1, \dots q_N, p_1, \dots p_N \}$
ocupa un ``mínimo volumen'' igual a $h^{3N}$.

\begin{enumerate}[label=(\alph*),
                  leftmargin=2\parindent,
                  rightmargin=2\parindent]

    \item{\label{item:de-broglie-particion}
          Obtener la función de partición del gas ideal aplicando la discretización del
          espacio de las fases.
          }

    \item{Definimos a la longitud de onda térmica de de Broglie como:
          $$ \lambda = \left( \frac{h^2}{2\pi m k_B T} \right)^{1/2} $$
          Expresar la función de partición obtenida en el punto
          anterior en función de $\lambda$.
          }

    \item{La longitud de onda de de Broglie se puede considerar como la
          longitud característica a partir de la cual la coherencia
          cuántica comienza a tener relevancia.\\
          Obtener una expresión para la distancia media de las
          partículas ($l$). ¿Cómo deberíamos considerar a las
          partículas si $\lambda/l \ll 1$? ¿y en el caso contrario?
          }

\end{enumerate}


\section{Paradoja de Gibbs}

Consideremos un sistema aislado compuesto por dos gases $A$ y $B$ en
equilibrio térmico que se encuentran separados por un tabique que
permite intercambiar energía entre ellos.
Los subsistemas $A$ y $B$ constan de $N_A$ y $N_B$ partículas del mismo
tipo y volúmenes $V_A$ y $V_B$, respectivamente.
Luego, retiramos el tabique permitiendo que ambos gases se mezclen.

\begin{figure}[h!]
    \centering
    \begin{subfigure}[b]{0.3\textwidth}
        \includegraphics[width=\textwidth]{figs/paradoja-gibbs-1.png}
        \caption{Estado inicial del sistema}
        \label{fig:gibbs-tabique}
    \end{subfigure}
    \hspace{0.1\textwidth}
    \begin{subfigure}[b]{0.3\textwidth}
        \includegraphics[width=\textwidth]{figs/paradoja-gibbs-2.png}
        \caption{Estado final del sistema}
        \label{fig:gibbs-mezcla}
    \end{subfigure}
    \caption{Paradoja de Gibbs en mezcla de gases}
\end{figure}

A partir de un análisis puramente termodinámico podemos afirmar que:

\begin{itemize}

\item{Si ambos gases poseen densidades diferentes
      $\left( \frac{N_A}{V_A} \neq \frac{N_B}{V_B} \right)$, entonces la
      mezcla de los gases es un proceso irreversible y por ende
      $ \Delta S > 0 $.
      }

\item{Si por el contrario, ambos gases poseen la misma densidad
      $\left( \frac{N_A}{V_A} = \frac{N_B}{V_B} \right)$, entonces
      el proceso es reversible y $ \Delta S = 0 $.
      }

\end{itemize}

A la hora de calcular la función de partición del gas ideal (como en los problemas
\ref{sec:gas-ideal} y \ref{sec:de-broglie}) no estamos considerando la
indistinguibilidad de las partículas.
Gibbs propuso dividir la función de partición por el factor $N!$ con el objeto de
eliminar del conteo aquellos estados en los que únicamente se permutan partículas:

$$ Z(N, T, V) = \frac{1}{N!} [Z_1(T, V)]^N $$

\begin{enumerate}[label=(\alph*),
                  leftmargin=2\parindent,
                  rightmargin=2\parindent]

    \item{Calcular la diferencia entre la entropía final e inicial del
          sistema sin aplicar la corrección $1/N!$.
          Analizar si el signo de la variación de la entropía
          concuerda con el análisis termodinámico para gases $A$ y $B$ de
          igual y de distinta densidad.}

    \item{Calcular la misma variación de entropía ahora con la
          corrección $1/N!$. ¿Dicha corrección resuelve la paradoja
          de Gibbs?}

\end{enumerate}


\section{Gas ideal clásico: Pozo de potencial infinito}

En los problemas anteriores hemos observado que un desarrollo mecánico
estadístico de gases ideales puramente clásicos posee problemas
conceptuales que deben resolverse a través de agregados provenientes
de razonamientos cuánticos.
Si bien una completa descripción cuántica del mismo involucra
simetrizaciones de la función de onda (que veremos más adelante), en
este problema intentaremos incorporar algunos aspectos mecánico
cuánticos.

Una forma de aproximar un gas ideal es considerarlo como un conjunto de
$N$ partículas cuánticas no interactuantes, cuya distancia con el resto
es lo suficientemente grande como para poder obviar la coherencia
cuántica. Dado que las partículas se encuentran encerradas en un
recipiente de volumen $V$, podemos considerar a cada partícula inmersa
en un pozo de potencial infinito de ancho $V^{1/3}$ en cada dirección
($x, y, z$).

Los niveles energía de una partícula en un pozo de potencial infinito
unidimensional vienen dados por:
$$ E_n = \frac{\hbar^2 \pi^2}{2m V^{2/3}} n^2, \quad n=1, 2, 3, \dots $$

\noindent Mientras que la misma partícula en un pozo de potencial infinito
tridimensional posee el siguiente espectro de energía:
$$ E_{n_x, n_y, n_z} = \frac{\hbar^2 \pi^2}{2m V^{2/3}} (n_x^2 + n_y^2 + n_z^2),
\quad n_x, n_y, n_z = 1, 2, 3, \dots $$


Además de suponer que las partículas no interactúan entre sí,
asumiremos que son indistinguibles, lo que nos permite expresar la
función de partición del sistema como:
$$ Z(N,T,V) = \frac{1}{N!} [Z_1(T, V)]^N $$
donde $Z_1$ es la función de partición de una sola partícula.


\begin{enumerate}[label=(\alph*),
                  leftmargin=2\parindent,
                  rightmargin=2\parindent]

    \item{Calcular la función de partición del sistema.}

    {\small
    \textbf{Ayuda:} Puede ser necesario aproximar la sumatoria
    discreta de los estados en una integral del tipo:
    $$ Z_1(T, V) =
    \int_0^\infty g(\epsilon) e^{-\epsilon/k_B T} d\epsilon $$
    donde $g(\epsilon)$ es la densidad de estados, es decir la cantidad de
    estados con energía entre $\epsilon$ y $\epsilon + d\epsilon$:
    $$ g(\epsilon) d\epsilon = N(\epsilon + d\epsilon) - N(\epsilon)
    \quad \Rightarrow \quad g(\epsilon) = \frac{d N(\epsilon)}{d\epsilon} $$
    donde $N(\epsilon)$ es la cantidad de estados con energía menor o
    igual a $\epsilon$.
    }

    \item{Comparar la función de partición obtenida en el punto
          anterior con la que hemos obtenido en el problema
          \ref{sec:de-broglie}\ref{item:de-broglie-particion}.
          Analizar cómo fueron introducidas las constantes de Planck
          en ambos casos. ¿A qué conclusión podemos arribar acerca
          de los gases ideales y un modelado puramente clásico de los
          mismos?
          }

\end{enumerate}


\section{Paramagnetismo de Langevin (clásico)}
\label{sec:paramagnetismo-langevin}

Como hemos visto en la práctica anterior, un paramagneto regular puede ser
modelado como un conjunto de $N$ momentos magnéticos situados en los
nodos de una red cristalina, los cuales interactúan únicamente por el campo
magnético externo $\textbf{H}$ (y no entre ellos).

En este problema realizaremos un desarrollo puramente clásico del
paramagneto, considerando que cada momento magnético es un vector de
módulo $\mu$ que puede orientarse en cualquier dirección dada por los
ángulos polar y azimutal $\theta$ y $\phi$, respectivamente.
Si el campo externo $\textbf{H}$ está orientado en la dirección del eje
$z$, entonces la energía de cada momento magnético viene dada por:

$$ E_i = - \boldsymbol{\mu_i} \cdot \textbf{H} = - \mu H \cos \theta_i $$


\begin{enumerate}[label=(\alph*),
                  leftmargin=2\parindent,
                  rightmargin=2\parindent]

    \item{Obtener la función de partición del paramagneto.}

    \item{Si definimos como $M$ a la proyección de la magnetización total del
          sistema en el eje $z$, demostrar que:
          $$ M(N, H, T) = - \left( \frac{\partial F}{\partial H}
          \right)_{N, T}$$
          }

    {\small
    \textbf{Ayuda:}
    Partir de la definición de la energía libre de Helmholtz ($ dF = dU
    - S dT$) y utilizar la expresión de la energía total del
    paramagneto ($ E = - \textbf{M} \cdot \textbf{H} \simeq - MH$).
    }

    \item{Obtener una expresión para la magnetización $M$ en función
          de la temperatura $T$ y realizar una gráfica de la misma.}

    {\small
    \textbf{Ayuda:} La función de Langevin $L(x)$ se define como:
    $$ L(x) = \coth x - \frac{1}{x} $$
    }

    \item{Verificar si la magnetización $M$ responde a la Ley de Curie
          para altas temperaturas.}

    \item{Otra forma de verificar la Ley de Curie es comprobar que la
          susceptibilidad magnética
          $\chi = \left( \frac{\partial M}{\partial H}\right)_{N, T}$
          es inversamente proporcional a $T$.
          Calcular la susceptibilidad $\chi$ y comprobar la Ley de
          Curie en el límite de altas $T$.
          }

    \item{Calcular el calor específico del paramagneto, graficarlo y
          analizar sus comportamientos a altas y bajas temperaturas.
          ¿Este resultado satisface el Tercer Principio de la
          Termodinámica?}


\end{enumerate}

\section{Paramagnetismo de Brillouin}

Abordemos ahora un paramagneto pero desde una perspectiva puramente
microscópica. Los átomos situados en los nodos de una red cristalina pueden
presentar electrones sin aparear debido a las Reglas de Hund. Al igual que
cualquier partícula elemental, los electrones presentan un momento angular
intrínseco llamado \emph{spin}, causal del momento magnético
$\boldsymbol{\mu}$.
Si suponemos que cada momento magnético interactúa únicamente con un
campo magnético externo $\textbf{H}$ (y no con el resto de los momentos
magnéticos), el hamiltoniano de cada uno puede escribirse como:
$$ \hat{\mathcal{H}} =
- \hat{\boldsymbol{\mu}} \cdot \textbf{H} = - H \hat{\mu_z} $$

Consideremos que cada átomo del paramagneto bajo estudio posee un solo
electrón sin aparear, por ende el operador $\hat{\mu_z}$ puede
escribirse como:
$$ \hat{\mu_z} = \frac{g_s \mu_B}{\hbar} \hat{S_z} $$

donde $\hat{S_z}$ es la componente $z$ del operador spin, $\mu_B$ el
magnetón de Bohr y $g_s \simeq 2$ el factor $g$ de spin. Ya que los
autovalores de $\hat{S_z}$ pueden ser $\pm 1/2 \hbar$, los niveles de
energía disponibles para cada momento magnético son:
$$ E_{m_s} = - g_s \mu_B H m_s, \quad  m_s = \pm 1/2.$$

\begin{enumerate}[label=(\alph*),
                  leftmargin=2\parindent,
                  rightmargin=2\parindent]

    \item{Obtener la función de partición del paramagneto.}

    \item{Obtener una expresión para la magnetización $M$ en función
          de la temperatura $T$ y realizar una gráfica de la misma.
          Comparar este modelo con el resuelto en la Práctica 1 a
          través del Microcanónico.
          ¿Ambos modelos son equivalentes?
          ¿Se obtienen los mismos resultados?}

    \item{Verificar si la magnetización $M$ responde a la Ley de Curie
          para altas temperaturas.}

    \item{Calcular el calor específico del paramagneto, graficarlo y
          analizar sus comportamientos a altas y bajas temperaturas.
          Compararlo con el calor específico obtenido en el
          Problema~ \ref{sec:paramagnetismo-langevin}.
          ¿A qué conclusión podemos arribar a partir de dicha comparación?
          }

\end{enumerate}


\section{Modelo de Sólido: Osciladores armónicos clásicos}
\label{sec:oscilador-clasico}

Consideremos el mismo modelo de sólido que propusimos en la Práctica 1:
un conjunto de $N$ partículas situadas en los nodos de una red
cristalina que oscilan alrededor de su posición de equilibrio en las 3
direcciones con la misma frecuencia $\omega$ y de manera independiente.

En este problema realizaremos un modelo puramente clásico del sólido,
considerándolo como un conjunto de $N$ osciladores armónicos clásicos.
La energía de cada uno de ellos viene dada por:

$$ E = \frac{1}{2m}(p_x^2 + p_y^2 + p_z^2) +
       \frac{1}{2} m \omega^2 (x^2 + y^2 + z^2) $$


\begin{enumerate}[label=(\alph*),
                  leftmargin=2\parindent,
                  rightmargin=2\parindent]

    \item{Calcular la función de partición de los $N$ osciladores.}

    \item{Calcular el calor específico de los mismos y comprobar que a
          altas temperaturas verifica la Ley de Dulong-Petit
          ($C_v = 3Nk_B$).}

    \item{¿Verifica este modelo el Tercer Principio de la
          Termodinámica? ¿A qué conclusión puede arribar con estos
          resultados?}

\end{enumerate}



\section{Sólido de Einstein}

Consideremos el mismo modelo de sólido del problema anterior (un
conjunto de $N$ partículas situadas en los nodos de una red cristalina
que oscilan alrededor de su posición de equilibrio en las 3 direcciones
con la misma frecuencia $\omega$ y de manera independiente), solo que
ahora trataremos a cada oscilador como un oscilador armónico
cuántico, cuya energía viene dada por:

$$
E_{n_x, n_y, n_z} = \left[
                    \left( n_x + \frac{1}{2} \right) +
                    \left( n_y + \frac{1}{2} \right) +
                    \left( n_z + \frac{1}{2} \right)
                    \right]
                    \hbar \omega,
\quad
n_x, n_y, n_z = 1, 2, 3, \dots
$$


\begin{enumerate}[label=(\alph*),
                  leftmargin=2\parindent,
                  rightmargin=2\parindent]

    \item{Calcular la función de partición del sólido.
          Obtener luego una expresión del valor medio de la energía y a
          partir de ella determinar su calor específico ($C_V$).}

    \item{Realizar una gráfica del calor específico del sólido en
          función de T.
          Verificar que se cumple la ley de Dulong y Petit para altas
          temperaturas ($C_V \simeq 3Nk_B$).}

    \item{Definamos la temperatura de Einstein como
          $T_E = \hbar \omega / k_B $, la cual es característica de
          cada material.
          Por ejemplo, el aluminio presenta una $T_E \sim 240 K$, el cual
          cumple la Ley de Dulong y Petit a temperatura ambiente.
          Sin embargo, existen excepciones a dicha ley, tal como el
          caso del diamante, el cual presenta una $T_E \sim 1320 K$.
          ¿Cómo podríamos aproximar el calor específico del diamante
          a temperatura ambiente?
          ¿Cómo se puede interpretar esta diferencia entre ambos
          materiales en función del modelo microscópico aplicado?
          }

    \item{Comparar el calor específico obtenido en el punto anterior con el del
          Problema~\ref{sec:oscilador-clasico}.
          Demostrar que el $C_V$ calculado satisface el Tercer Principio.
          ¿Cómo podemos interpretar este resultado?}

    \item{La energía total del sólido puede expresarse como:
          $$ E = \sum_{i=1}^{3N} \left( n_i + \frac{1}{2} \right) \hbar \omega, $$
          donde los $n_i$ son los niveles de ocupación de los
          osciladores unidimensionales, es decir, el nivel energético
          que ocupa cada oscilador.
          Si calculamos el valor medio de la energía del sólido, la
          expresión anterior puede reescribirse como:
          $$ \langle E \rangle =
          \sum_{i=1}^{3N}
          \left( \langle n_i \rangle + \frac{1}{2} \right) \hbar \omega, $$
          y dado que los osciladores son independientes, el valor medio
          del nivel de ocupación será el mismo para todos:
          $\langle n \rangle$.
          Por lo tanto, podemos escribir el valor medio de la energía
          del sólido de la siguiente manera:
          $$ \langle E \rangle =
          3N \left( \langle n \rangle + \frac{1}{2} \right) \hbar \omega. $$
          Demostrar que el valor medio del nivel de ocupación puede
          expresarse como:
          $$ \langle n \rangle = \frac{1}{e^{\hbar \omega / kT} - 1}. $$
          Obtener los valores medios del nivel de ocupación en caso de
          que $ T = T_E/2$, $T= T_E$ y $T= 3T_E$.}

\end{enumerate}


\end{document}

